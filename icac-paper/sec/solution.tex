This section describes two different solutions for balancing the
load directed to self-adaptive brownout-compliant applications
composed of multiple replicas and their implementation

\subsection{Heuristics design}

Our first solution is referred in this paper as \textbf{Variational
  Principle Based} and inspired by the predictive approach described,
in Section~\ref{sec:Related}. The core of the predictive solution is
to examine the variation of the involved quantities. While in its
classical form, this solution relies on variations of response times
or pending request count per replica, our solution is based on the how
the control variables $\theta_i$ are changing. If the percentage
$\theta_i$ of optional content served is increasing, the replica is
assumed to be less loaded, and more traffic can be sent to it. On the
contrary, when the optional content decreases, the replica will
receive less traffic, to decrease its load and allow it to increase
$\theta_i$.

The replica weights $w_i$ are initialized to $\frac{1}{n}$ where $n$
is the number of replicas. The load-balancer periodically updates the
values of the weights based on the values of $\theta_i$ received by
the replicas. At time $k$, denoting with $\Delta \theta_i (k)$ the
variation $\theta_i (k) - \theta_i (k-1)$, the solution computes a
potential weight $\tilde{w}_i(k+1)$ according to
\begin{equation}
  \tilde{w}_i(k+1) = w_i(k) \cdot 
  \left[ 1 + \gamma_p \, \Delta \theta_i (k) 
    + \gamma_i \, \theta_i (k) \right] ,
\label{eq:theta-diff}
\end{equation}
where $\gamma_p$ and $\gamma_i$ are constant gains, respectively
related to a proportional and an integral load-balancing action. As
calculated, $\tilde{w}_i$ values can be negative. This is clearly not
feasible, therefore negative values are truncated to a small but still
positive weight. Using a positive weight instead of zero allows us to
probe the replica and see whether it is favorably responding to new
incoming requests or not. Moreover, the computed values do not respect
the constraint that their sum is equal to 1, so they are then
re-scaled according to
\begin{equation}
  w_i (k) = \cfrac{\tilde{w}_i (k)}{\sum_i \tilde{w}_i (k)}.
\label{eq:theta-diff-rescale}
\end{equation}

\begin{itemize}
\item Selection of parameters.
\end{itemize}

The second self-aware policy that we present is called
\textbf{Equality Principle Based} and relies on the heuristic
principle that in the best conditions, every replica should have a
similar behavior, therefore the control variables $\theta_i$ should be
as close as possible to one another. In fact, if the values of
$\theta_i$ converge to a single value, this means that the traffic is
routed so that each replica can serve the same percentage of optional
content, i.e., the most powerful replica is receiving more traffic
with respect to the least powerful one. The distribution should
ideally allow to converge to a value that is maximum with the entire
pool of requests received by the application. This approach therefore
selects weights that would encourage the control variables $\theta_i$
to converge to a single value.

The policy computes a potential weight $\tilde{w}_i(k+1)$
\begin{equation}
  \tilde{w}_i(k+1) = w_i(k) + \gamma_e e_i(k)
\label{eq:equal-thetas}
\end{equation}
where
$$e_i(k)=\left[ \theta_i (k) - \cfrac{\sum_j \theta_j (k) }{n} \right]$$
and $\gamma_e$ is a non-zero positive parameter of the algorithm which
accounts for how fast the algorithm should converge. The weights are
simply modified proportionally to the difference between the current
control value and the average control value set by the
replicas. Clearly, the same saturation and normalization described in
Equation \eqref{eq:theta-diff-rescale} have to be applied to the
proposed solution, to ensure that the sum of the weights is equal to
one and that they have positive values --- i.e., that all the incoming
traffic is directed to the replicas and that each replica receives at
least some requests.

\begin{itemize}
\item Selection of paramter.
\end{itemize}

\subsection{Implementation}

\begin{itemize}
\item Describe the implementation.
\end{itemize}