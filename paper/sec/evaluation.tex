\textcolor{blue}{\textit{\textbf{Martina:} How to evaluate our
    strategies:
    \begin{itemize}
    \item percentage of requests served with optional code turned on
    \item standard deviations of response times: they call it user-perceived-stability~\cite{GeograficalSASO}
    \item more?
    \end{itemize}
}}.

\subsection{Simulator}

\textcolor{blue}{\textit{\textbf{Martina:} We should here describe our
    simulator}}.

\subsection{Extended case study}

\textcolor{blue}{\textit{\textbf{Martina:} Here we should have one
    extended case study that we describe properly. A certain amount of
    time where a few things happen --- new clients entering the
    system, old clients abandoning the system and it would also be
    nice to change the execution time for mandatory and optional code
    (in this last way we can simulate the fact that we have less cores
    or more cores available and this is changed by the infrastructure
    provider without our control). We should demonstrate that in this
    extended case study we do better than some existing load-balancing
    policies. Any idea on policies that we should simulate?}}.

\textcolor{red}{\textit{\textbf{Cristian:} Also, we should test the
    load-balancer when the number of replicas changes.}}

\subsection{Aggregate case study}

\textcolor{blue}{\textit{\textbf{Martina:} We can run some simulations
    and presents some aggregate indexes instead of time series --- for
    example the total percentage of recommendations served by our
    strategy and existing ones and the maximum response time through
    the simulation lenght with our and existing strategy. Ideally the
    subsection above should have graphs and this one should result in
    a table with a lot of different case studies.}}.



