In this section we describe our experimental evaluation, discussing
the performance indicators used to compare different strategies, the
simulator developed and used to emulate the behavior of
brownout-compliant replicas driven by the load-balancer and our case
studies.

\subsection{Performance indicators}

Performance measures are necessary to objectively compare different
algorithms. While our first performance indicator is clearly defined
as the \textbf{percentage $\%_{on}$} of the total requests served with
the optional content enabled, we also would like to introduce some
other performance metrics to compare the implemented load-balancing
techniques.

Another metric considered for this study is the \textbf{user-perceived
  stability $\sigma_u$}~\cite{GeograficalSASO}. This metric refers to
the variation of perfomance as observed by the users, and it is
measured as the standard deviation of the vector of response
times. Its purpose is to measure the ability of the replicas to
respond timely to the client requests. The entire brownout framework
aims at stabilizing the response times, therefore it should achieve
low user-perceived stability, irregardless of the presence of the
load-balancer. However, the load-balancing algorithm clearly
influences the perceived latencies, therefore it is logical to check
whether the newly developed algorithms achieve a better perceived
stability with respect to the classical ones. Together with the value
of the user-perceived stability, we also report the \textbf{average
  response time $\mu_u$} to distinguish between algorithms that
achieve a low response time with possibly high fluctuations from
solutions that achieve a higher but more stable response time.

\str{Do we want to add more performance indicators?}

\subsection{Simulator}

\str{Describe the simulator}

\str{State that implementing new algorithms is easy}

\str{Stress the fact that the scenario-based approach makes the
  simulator very flexible and it is therefore very easy to support
  multiple test cases}

\subsection{Reacting to clients behavior}

\begin{itemize}
\item \str{Clients entering at different times}
\item \str{Clients leaving the system at some point}
\end{itemize}

\subsection{Reacting to infrastructure resources}

\begin{itemize}
\item \str{Change execution time for mandatory and optional, therefore
    simulating the change in resources provided to the replicas}
\item \str{Change the number of replicas}
\end{itemize}

