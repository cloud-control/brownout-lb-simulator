In this section we describe our experimental evaluation, discussing
the performance indicators used to compare different strategies, the
simulator developed and used to emulate the behavior of
brownout-compliant replicas driven by the load-balancer and our case
studies.

\subsection{Performance indicators}

Performance measures are necessary to objectively compare different
algorithms. While our first performance indicator is clearly defined
as the \textbf{percentage $\%_{on}$} of the total requests served with
the optional content enabled, we also would like to introduce some
other performance metrics to compare the implemented load-balancing
techniques.

Another metric considered for this study is the \textbf{user-perceived
  stability $\sigma_u$}~\cite{GeograficalSASO}. This metric refers to
the variation of perfomance as observed by the users, and it is
measured as the standard deviation of the vector of response
times. Its purpose is to measure the ability of the replicas to
respond timely to the client requests. The entire brownout framework
aims at stabilizing the response times, therefore it should achieve
low user-perceived stability, irregardless of the presence of the
load-balancer. However, the load-balancing algorithm clearly
influences the perceived latencies, therefore it is logical to check
whether the newly developed algorithms achieve a better perceived
stability with respect to the classical ones.

\subsection{Simulator}

\textcolor{blue}{\textit{\textbf{Martina:} We should here describe our
    simulator}}.

\subsection{Extended case study}

\textcolor{blue}{\textit{\textbf{Martina:} Here we should have one
    extended case study that we describe properly. A certain amount of
    time where a few things happen --- new clients entering the
    system, old clients abandoning the system and it would also be
    nice to change the execution time for mandatory and optional code
    (in this last way we can simulate the fact that we have less cores
    or more cores available and this is changed by the infrastructure
    provider without our control). We should demonstrate that in this
    extended case study we do better than some existing load-balancing
    policies. Any idea on policies that we should simulate?}}.

\textcolor{red}{\textit{\textbf{Cristian:} Also, we should test the
    load-balancer when the number of replicas changes.}}

\subsection{Aggregate case study}

\textcolor{blue}{\textit{\textbf{Martina:} We can run some simulations
    and presents some aggregate indexes instead of time series --- for
    example the total percentage of recommendations served by our
    strategy and existing ones and the maximum response time through
    the simulation lenght with our and existing strategy. Ideally the
    subsection above should have graphs and this one should result in
    a table with a lot of different case studies.}}.



