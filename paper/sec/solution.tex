This section describe three different solutions for balancing the load
directed to self-adaptive brownout-compliant applications composed by
multiple replicas. The first two strategies are heuristic solution
that take into account the self-adaptivity of the replicas. The third
alternative is based on optimization theory, with the aim of providing
guarantees on the best possible behavior.

\subsection{Variational principle-based heuristic}

Our first solution is inspired by the predictive approach. In fact,
the core of the predictive approach is to examine the variation of the
involved quantities. Instead of examining the variations of the
response times or of the number of connections per replica, the
approach is based on the how the control variables $\theta_i$ are
changing within each replica.

The replica weights $w_i$ are initialized to $\frac{1}{n}$ where $n$
is the number of replicas. The load-balancer periodically updates the
values of the weights based on the values of $\theta$ received by the
replicas. At time $k$, denoting with $\Delta \theta_i (k)$ the
variation $\theta_i (k) - \theta_i (k-1)$, the new weight $w_i (k)$ is
computed as
\begin{equation}
  w_i(k) = \cfrac{w_i (k-1) + \gamma \cdot \Delta \theta_i (k)}
                 {\sum_j w_j (k-1) + \gamma \cdot \Delta \theta_j (k)}
\label{eq:theta-diff}
\end{equation}
where $\gamma$ is simply a fixed gain that takes into account the
relative difference between the variation of the control variable and
the weights, in our experiments its value is set to $0.5$. The
approach determines the variation of the control variables $\theta_i$
and corresponingly react. If the percentage of optional content served
is increasing, the replica is assumed to be less loaded, and more
traffic can be sent to it. On the contrary, when the optional content
diminish, the replica will receive less traffic, to increase the
control variable if possible.

As stated in Equation~\eqref{eq:theta-diff}, the approach has one main
drawback. If the value of $\theta_i$ is constant and equal to $1$ its
variation $\Delta \theta_i$ is equal to zero. However, the replica
could possibly serve more requests, since the optional computations
are always enabled. Therefore, in case we detect such a condition, we
force $\Delta \theta_i$ to be equal to $0.01$ to indicate that the
approach should probe the replica and try to send it more requests. In
case the value of $\theta_i$ is constant and equal to zero, the
approach should send less requests to the replica, so no adjustments
are made for such a case.

\subsection{Equal dimmers}
\str{Find a name for it and describe it. If we can prove something
  about the solution, add the proof here.}

\subsection{Optimization based load-balancing}
\str{Find a name for it and describe it. Prove what we can prove about
  the solution.}

% \textcolor{red}{\textit{\textbf{Cristian:} Some stuff that might help
%     to devise a solution. It can be proven (i.e., I have the proof on
%     a piece of paper), that the effective service rate of a replica
%     is: $$\mu_{i,eff}=\frac{1}{\frac{1-\theta_i}{\mu_i}+\frac{\theta_i}{M_i}}$$
%     \\
%     The respose time of a replica, as per M/M/1 queues, is:
%     \\
%     $$t_i=\frac{1}{\mu_{i,eff}-\lambda_i}$$
%     \\
%     Knowing that the controller tries to maintain $t_i=\tau_i$, one
%     can compute $\theta_i$ by inverting the equation above.
% }}
	
% \textbf{Proof for effective service rate.} Let $x_i = 1/\mu_i$ be the service time of a request served with mandatory content and $X_i = 1/M_i$ be the service time of a request served with both mandatory and optional content. Let us compute the time required to serve a burst of $n$ requests, i.e., $n$ requests that arrive at the same time, with a dimmer value of $\theta_i$:

% $$t=n (1-\theta_i) x_i + n \theta_i X_i$$

% Hence, the effective service rate is:

% \begin{eqnarray*}
% \mu_{eff,i} & = & n/t \\
% & = & \frac{1}{(1-\theta_i) x_i + \theta_i X_i} \\
% & = & \frac{1}{\frac{1-\theta_i}{\mu_i} + \frac{\theta_i}{M_i}}
% \end{eqnarray*}

