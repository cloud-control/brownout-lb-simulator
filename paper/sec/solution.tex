\textcolor{blue}{\textit{\textbf{Martina:} We can describe here the
    two different solutions. One the one based on the relative
    difference between dimmer values we should work a little bit more
    to avoid the undesired behavior that if the dimmer stays to the
    highest value we stop sending stuff. I will take care of that and
    of describing that solution here.}}

\textcolor{red}{\textit{\textbf{Cristian:} Some stuff that might help
    to devise a solution. It can be proven (i.e., I have the proof on
    a piece of paper), that the effective service rate of a replica
    is: $$\mu_{i,eff}=\frac{1}{\frac{1-\theta_i}{\mu_i}+\frac{\theta_i}{M_i}}$$
    \\
    The respose time of a replica, as per M/M/1 queues, is:
    \\
    $$t_i=\frac{1}{\mu_{i,eff}-\lambda_i}$$
    \\
    Knowing that the controller tries to maintain $t_i=\tau_i$, one
    can compute $\theta_i$ by inverting the equation above.
}}
	
\textbf{Proof for effective service rate.} Let $x_i = 1/\mu_i$ be the service time of a request served with mandatory content and $X_i = 1/M_i$ be the service time of a request served with both mandatory and optional content. Let us compute the time required to serve a burst of $n$ requests, i.e., $n$ requests that arrive at the same time, with a dimmer value of $\theta_i$:

$$t=n (1-\theta_i) x_i + n \theta_i X_i$$

Hence, the effective service rate is:

\begin{eqnarray*}
\mu_{eff,i} & = & n/t \\
& = & \frac{1}{(1-\theta_i) x_i + \theta_i X_i} \\
& = & \frac{1}{\frac{1-\theta_i}{\mu_i} + \frac{\theta_i}{M_i}}
\end{eqnarray*}

