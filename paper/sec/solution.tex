This section describe three different solutions for balancing the load
directed to self-adaptive brownout-compliant applications composed by
multiple replicas. The first two strategies are heuristic solution
that take into account the self-adaptivity of the replicas. The third
alternative is based on optimization theory, with the aim of providing
guarantees on the best possible behavior.

\subsection{Variational principle-based heuristic}

Our first solution is inspired by the predictive approach. In fact,
the core of the predictive approach is to examine the variation of the
involved quantities. Instead of examining the variations of the
response times or of the number of connections per replica, the
approach is based on the variations of the control variable $\theta$.

\subsection{Equal dimmers}
\str{Find a name for it and describe it. If we can prove something
  about the solution, add the proof here.}

\subsection{Optimization based load-balancing}

\textcolor{red}{\textit{\textbf{Cristian:} Some stuff that might help
    to devise a solution. It can be proven (i.e., I have the proof on
    a piece of paper), that the effective service rate of a replica
    is: $$\mu_{i,eff}=\frac{1}{\frac{1-\theta_i}{\mu_i}+\frac{\theta_i}{M_i}}$$
    \\
    The respose time of a replica, as per M/M/1 queues, is:
    \\
    $$t_i=\frac{1}{\mu_{i,eff}-\lambda_i}$$
    \\
    Knowing that the controller tries to maintain $t_i=\tau_i$, one
    can compute $\theta_i$ by inverting the equation above.
}}
	
\textbf{Proof for effective service rate.} Let $x_i = 1/\mu_i$ be the service time of a request served with mandatory content and $X_i = 1/M_i$ be the service time of a request served with both mandatory and optional content. Let us compute the time required to serve a burst of $n$ requests, i.e., $n$ requests that arrive at the same time, with a dimmer value of $\theta_i$:

$$t=n (1-\theta_i) x_i + n \theta_i X_i$$

Hence, the effective service rate is:

\begin{eqnarray*}
\mu_{eff,i} & = & n/t \\
& = & \frac{1}{(1-\theta_i) x_i + \theta_i X_i} \\
& = & \frac{1}{\frac{1-\theta_i}{\mu_i} + \frac{\theta_i}{M_i}}
\end{eqnarray*}

