Load-balancing is a standard component of Internet-scale services,
allowing them to achieve scalability and
resilience~\citep{Barroso09,Hamilton07:LISA,clusteredbalancing}.  Many
load-balancing policies have been proposed, aiming at different
optimizations, spanning from processor load~\cite{Stankovic:TC} to
memory management~\cite{PattersonMemoryLB,MemoryLBACC} and web-server
systems~\cite{Cardellini2003}. Load-balancing strategies can be guided
by many different aims, for example
geografical~\cite{GeograficalSASO,geographicalwanbalancing} or driven
by the electricity price was proposed to reduce the datacenter
operation cost~\cite{LoadBalancingForElectricity:TCC}. Cloud
applications have given a further boost to the
research~\citep{Barroso09,Lu11:PerfEval,Lin12:IGCC}.

\str{Reinforce citations for the static load-balancing.}

Load-balancing solutions can be divided into two different types:
static and dynamic policies. Static load-balancing refers to a fixed
strategy to select the next replica to be loaded. Static
load-balancing found many applications in load sharing on
multiprocessor
systems~\cite{StaticLoadBalancing:TSE,StaticOptimal:ACM}. Among the
static load-balancing techniques the most commonly implemented are
random balancing, and \textbf{Round Robin (RR)} based ones. Random
balancing is based on random decisions, and despite being extremely
simple to implement, his utility is questionable, since it does not
take into account any factor in selecting the replica. Round robin
solutions try to balance the load onto the replicas. A round robin
algorithm selects the replica, eventually weighting the possible
alternatives based on a priori knowledge of their capabilities. In
this case the algorithm is often referred to as \textbf{Weighted RR}.

Although in our comparison we use also a weighted round robin
solution, in principle we are not interested in static
load-balancing. In fact, the purpose of brownout-compliant
applications and load-balancing is to account for the inherent
fluctuations of a cloud environment. The same replica can receive
different resources during its lifetime, therefore a static weight
vector would not reflect the current situation.

On the contrary, in dynamic load-balancing the choice of the replica
is based on measurements about the current system state. One widely
used option is the choice of the fastest replica, therefore realizing
a \textbf{Fastest Replica First (FRF)} load balancer.
\str{Describe FRF-EWMA here.}

FRF and its variations are unlikely to work with brownout-compliant
applications, because each replica selects the amount of optional code
executed to maintain the response times to a certain desired value,
and there is no way to distinguish between a congested replica that is
not serving optional content and a non-congested replica that could
serve the optional content. Also with non-adaptive applications this
algorithm can cause severe performance loss, since the current
response time does not necessarily reflect the future response time
and the internal replica queue can saturate at an unexpected rate,
especially if the time taken to produce the answer is much greater
than the arrival rate.

Another adopted strategy is based on the connection count and
generally called \textbf{Shortest Queue First (SQF)}, where the
load-balancer tracks the connections and select the replicas with the
least amount of connections. This strategy pays off in architectures
where the replicas have similar capacities and the requests are
homogeneous. To account for non-homongeneity, Pao and Chen proposed a
load balancing solution using the remaining capacity of the replicas
to determine how the next request should be
managed~\cite{feedbackintensive}. The capacity is determined through a
combination of factors like the remaining available CPU and memory,
the network transmission and the current connection count.  These
strategies may not be optimal if the requests can be very different
from one another --- for example a request serves a plain HTML page
while another requires looking up a lot of data in a database. This
remark also suggests that the strategy is unlikely to work with
brownout-compliant applications, where requests served with optional
computations can be very different from the ones containing only the
mandatory content.

Stankovic proposed a Bayesian approach to the load-balancing problem
for a multiprocessor load distribution~\cite{Stankovic:TC}. An
estimator of the queue length per processor is kept updated and the
next request is sent to the least congested computing unit. This
approach is hardly applicable to brownout-compliant cloud applications
because it does not take into account the replica control loop
adjusting the percentage of optional code executed. Since in our case
the load-balancer is the only entity routing requests to the replica,
this approach is equivalent to the connection count policy. A
variation of this algorithm is the \textbf{Two Random Choices
  (2RC)}~\cite{2RC} one, that randomly chooses two replicas and
assigns the request to the fastest one, i.e., the one with the
lowest maximum latency.

In an attempt to overcome the mentioned limitations a merge of the
fastest replica and the connection count approach was implemented
where the replicas are ranked based on a linear combination of
response times and number of routed requests. Also, the approach was
extended with a \textbf{Predictive} estimator. To produce the ranking
the difference between the past metrics and the current one was
used. This is the solution implemented in the BIG-IP Local Traffic
Manager~\cite{BIGIP}. One of the solutions proposed in this paper
borrows the idea of looking at the difference between the past
behavior and the current one, although our load-balancer observes the
changes in the amount of optional code served and tries to maximize
the requests served enabling the full computation.

Dynamic solutions can be based on control
theory~\cite{multipathctlb,comparisonstaticdynamic} and also account
for the cost of applying the control action~\cite{costofcontrol}. This
is especially necessary where the load balancer acts also as a
resource allocator and the induced sudden lack of the resource can
result in poor performances. This is not an issue with the stateless
requests that are typical in the cloud domain.

