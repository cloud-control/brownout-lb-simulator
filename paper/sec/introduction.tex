Cloud computing is expected to be one of the driving force of the
future economy~\citep{WPonMckinsey13}. Already, it has dramatically
changed the management of computing infrastructures. On one hand,
public infrastructure providers, such as Amazon EC2, allow service
providers to deploy their services on large infrastructures with no
up-front cost~\citep{Buyya09:FGCS}. On the other hand, the flexibility
offered by cloud technologies themselves favors the adoption of
private clouds~\citep{Gulati11:HotCloud}, therefore, self-hosting
service providers themselves are converting their computing
infrastructures into small, internally-managed clouds.

One of the main advantages offered by cloud infrastructures is {\bf
  elasticity}. A service provider that wants to deploy a new service
on the cloud can rapidly provision the required computing resources,
automatically acquiring and releasing computing resources as
required~\citep{Herbst13:ICAC}.  Elasticity can be of two
complementary types: vertical and horizontal. Vertical elasticity
consists in adding or removing resources (e.g., CPU cores) from an
existing \ac{vm}, while horizontal elasticity deals with changing the
number of \acp{vm} allocated to a specific service, adding a new
machine or removing an existing one.
%
Horizontal elasticity calls for the introduction of a specialized
component, called {\bf load-balancer}, that takes care of routing the
end-user requests to one of the \acp{vm} composing the
service. Load-balancing techniques have been widely studied and
adopted~\citep{Barroso09,Lu11:PerfEval,Lin12:IGCC,BeesBased:ADAPTIVE}.

One of the main issues with cloud computing infrastructures, however,
is {\bf robustness} to unexpected events. For example, flash-crowds
are sudden increases of end-users, that may increase the required
capacity by up to five times~\citep{Bodik10:SoCC}. Similarly, hardware
failures may temporarily reduce the capacity of the infrastructure,
while the failure is repaired~\citep{Barroso09}. Due to the large
magnitude and short duration of such events, it may be economically
too costly to keep enough spare capacity to properly deal with
them. As a result, unexpected events may lead to infrastructure
overload, which translates to unresponsive services, leading to
dissatisfied end-users and revenue loss.

Cloud services therefore greatly benefit from self-adaptation
techniques~\cite{SalehieSelfadaptive:TAAS}, such as {\bf
  brownout}~\citep{cloudish-tr}. A brownout service adapts itself by
reducing the amount of computations it executes to produce a response,
so as to maintain response time around a given setpoint.  In essence,
some computations are marked as mandatory --- for example, the display
of the product information in an e-commerce website --- while others
are optional --- for example, the display of recommendations of
similar products.  Whenever an end-user request is received, the
service can choose to execute or not the optional code based on the
capacity available to it, to which it adapts by monitoring response
times. Note that executing optional code directly translates into a
better service for the end-user and more revenues for the service
provider. This approach proved to be successful for dealing with
unexpected events~\citep{cloudish-tr}. However, brownout services were
composed of a single replica, running inside a single \ac{vm}.

In this paper, we extend the brownout paradigm to services featuring
multiple replicas, hosted inside individual \acp{vm}. We develop a
load-balancer that takes into account the replica adaptation to decide
how to forward requests to replicas.  Besides enabling horizontal
elasticity, having a load-balancer and multiple replicas makes
services fault-tolerant. Existing, state-of-the-art load-balancers
forward requests based on metrics that cannot discriminate between a
replica that is avoiding overload by not executing the optional code
and a replica that is not threatened with overload executing all
optional code. Therefore, existing load-balancing techniques must be
improved to migrate load away from replicas not executing optional
code.

Our challenge is to find a load-balancing methodology that maximizes
the amount of served optional content, provided that every replica
independently serves only an amount of optional content that allows it
to maintain the maximum response time below a configured target.

Our contribution is summarized as follows.
\begin{itemize}
\item We discuss load-balancing architectures and the required
  enhancements to the replicas that allow to deal with brownout
  services efficiently.
\item We propose load-balancing policies that aim at maximizing the
  performance of brownout services, in terms of frequency of execution
  of the optional code, therefore maximizing the total revenue for the
  service provider.
\item We evaluate existing load-balancing strategies and compare them
  with the proposed approaches, demonstrating the need for a
  brownout-aware load-balancer and showing that the proposed policies
  outperform existing techniques.
\end{itemize}
