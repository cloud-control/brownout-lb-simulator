Cloud computing is expected to be one of the driving force of the
future economy~\citep{WPonMckinsey13}. Already, it has dramatically
changed the management of computing infrastructures. On one hand,
public cloud providers, such as Amazon EC2, allow companies to deploy
their services on large infrastructures with no up-front
cost~\citep{Buyya09:FGCS}. On the other hand, the flexibility offered
by cloud technologies themselves favors the adoption of private
clouds~\citep{Gulati11:HotCloud}, therefore, companies are converting
their own computing infrastructures into small, internally-managed
clouds.

One of the main advantages offered by cloud infrastructures is {\bf
  elasticity}. A company that wants to deploy a new service on the
cloud can rapidly provision the required computing resources,
automatically acquiring and releasing \acp{vm} as
required~\citep{Herbst13:ICAC}.  Elasticity can be of two
complementary types: vertical and horizontal. Vertical elasticity
consists in adding or removing resources (e.g., CPU cores) from an
existing \ac{vm}, while horizontal elasticity deals with changing the
number of \acp{vm} allocated to a specific service, adding a new
machine or removing an existing one.
%
Horizontal elasticity calls for the introduction of a specialized
component, called {\bf load-balancer}, that takes care of routing the
requests to one of the \acp{vm} composing the service. Load-balancing
techniques have been widely studied and
adopted~\citep{Barroso09,Lu11:PerfEval,Lin12:IGCC}.

One of the main issues with cloud computing infrastructures, however,
is {\bf robustness} to unexpected events. For example, flash-crowds
are sudden increases of end-users, that may increase the required
capacity by up to five times~\citep{Bodik10:SoCC}. Similarly, hardware
failures may temporarily reduce the capacity of the data-center, while
the failure is repaired~\citep{Barroso09}. Due to the large magnitude
and short duration of such events, it may be economically too costly
to provision enough hardware to properly deal with them. As a results,
unexpected events may lead to data-center overload, which translates
to unresponsive cloud applications, leading to dissatisfied end-users
and revenue loss.

In our previous work~\citep{cloudish-tr}, we proposed a cloud
application development paradigm called {\bf brownout}. The main
characteristic of cloud brownout is to be able to cope with unexpected
events by reducing the amount of computation executed inside the
\ac{vm} to produce the response. In brownout applications some
computations are marked as mandatory --- for example, the display of the
product information in an e-commerce website --- others are optional ---
for example, the display of recommendations of similar products.
Whenever a request is
received, the application can choose to execute or not the optional
code based on the data-center conditions and on its own
optimization. Note that executing optional code directly translates
into a better service for the user and more revenues for the company. This approach
proved to be successful for dealing with unexpected events. However,
we only considered services having a single replica and running inside
a single \ac{vm}.

In this paper, we extend our previous work to multiple \acp{vm},
developing a brownout-compliant load-balancer. This enables horizontal
elasticity and makes applications fault-tolerant by having several
replicas. Existing, state-of-the-art load-balancers forward requests
based on the response-time of each replica, which leads to inefficient
decisions for brownout-compliant applications, since such applications
already keep their response-time at a given setpoint (at the expense
of reducing the amount of optional content served). Specifically, a
brownout-compliant load-balancer is necessary, since by measuring
response-time alone, it is not possible to discriminate between a
replica that is avoiding overload by not executing the optional code
and a replica that is not threatened with overload executing all
optional code, both achieving low response times.  Therefore, existing
load-balancing techniques that monitor response times should be
improved to migrate load away from replicas not executing optional
code.

Our challenge is to find a load-balancing methodology that maximizes
the amount of served optional content. Our contribution is threefold.
\begin{enumerate}
% According to the current sec:problem, the architecture is needed to formally define the problem.
\item We propose a load-balancing architecture for brownout-compliant
  cloud applications and formalize our problem
  (Section~\ref{sec:problem}).
\item We propose a load-balancing algorithm that maximizes the
  performance of brownout-compliant applications in terms of
  percentage of optional code executed (Section~\ref{sec:solution}).
\item We evaluate the approach showing that it outperforms existing
  techniques (Section~\ref{sec:evaluation}).
\end{enumerate}

