Cloud computing is expected to be the driving force of future IT innovation~\cite{Gartner}. Instead of forcing businesses to make costly and risky investment in IT infrastructure, public clouds, such as Amazon EC2, allow such businesses to rent computing infrastructure as they grow, with no upfront costs~\cite{Buyya}. In fact, the model proved so successful, that companies are now converting their IT infrastructure to so-called private clouds, i.e., data-centers managed using cloud technologies, to improve IT efficiency.



Thanks to technologies, such as virtualization, users can simply wrap their applications 


- applications are moving into the cloud, argue why, advantages, buzz, etc.
- virtual machines, capacities, how resources are allocated
- cloud infrastructure management: horizontal scaling, vertical scaling
- cloud to sustain unexpected events such as flash-crowds, hardware failures etc.
- previous contribution: brownout
- restricted to a single physical machine, vertical elasticity
- remaining issue: horizontal elasticity, redundancy through replication
- load balancer
- existing approaches use load or response-time, which is not okey for brownout-compliant applications
  - intuitively, since the application already controls its response-time at the expense of the number of optional requests served, existing load-balancing architectures and approaches cannot detect which replica is performing better.
- The contribution of this article is threefold:
  - We propose an architecture and a problem statement
  - We propose a control-theoretical load-balacing algorithm
  - We evaluate the approach

Problem statement
- architecture
- formalization
- question

