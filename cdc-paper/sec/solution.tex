This section describes three different solutions for balancing the
load directed to self-adaptive brownout-compliant applications
composed of multiple replicas. The first two strategies are heuristic
solutions that take into account the self-adaptivity of the
replicas. The third alternative is based on optimization theory, with
the aim of providing guarantees on the best possible behavior. In all
solutions, the load-balancer is configured with a control period of
$1$s.

\subsection{Variational principle-based heuristic (VPBH)}

Our first solution is inspired by the predictive approach described,
in Section~\ref{sec:related}. The core of the predictive solution is
to examine the variation of the involved quantities. While in its
classical form, this solution relies on variations of response times
or pending request count per replica, our solution is based on the how
the control variables $\theta_i$ are changing.

If the percentage $\theta_i$ of optional content served is increasing,
the replica is assumed to be less loaded, and more traffic can be sent
to it. On the contrary, when the optional content decreases, the
replica will receive less traffic, to decrease its load and allow it
to increase $\theta_i$.

The replica weights $w_i$ are initialized to $\frac{1}{n}$ where $n$
is the number of replicas. The load-balancer periodically updates the
values of the weights based on the values of $\theta_i$ received by
the replicas. At time $k$, denoting with $\Delta \theta_i (k)$ the
variation $\theta_i (k) - \theta_i (k-1)$, the solution computes a
potential weight $\tilde{w}_i(k+1)$ according to
\begin{equation}
  \tilde{w}_i(k+1) = w_i(k) \cdot 
\left[ 1 + \gamma_p \, \Delta \theta_i (k) + \gamma_i \, \theta_i (k) \right] ,
\label{eq:theta-diff}
\end{equation}
where $\gamma_p$ and $\gamma_i$ are constant gains, respectively
related to a proportional and an integral load-balancing action. As
calculated, $\tilde{w}_i$ values can be negative. This is clearly not
feasible, therefore negative values are truncated to a small but still
positive weight. Using a positive weight instead of zero allows us to
probe the replica and see whether it is favorably responding to new
incoming requests or not. Moreover, the computed values do not respect
the constraint that their sum is equal to 1, so they are then
re-scaled according to
\begin{equation}
  w_i (k) = \cfrac{\tilde{w}_i (k)}{\sum_i \tilde{w}_i (k)}.
\label{eq:theta-diff-rescale}
\end{equation}

We selected $\gamma_p = 0.5$ based on experimental results. Once
$\gamma_p$ is fixed to a selected value, increasing the integral gain
$\gamma_i$ calls for a stronger action on the load-balancing side,
which means that the load-balancer would take decisions very much
influenced by the current values of $\theta_i$, therefore greatly
improving performance at the cost of a more nervous control action. On
the contrary, decreasing $\gamma_i$ would stabilize the control
signal, possibly resulting in performance loss due to a slower
reaction time. The choice of the integral gain allows to exploit the
trade-off between performance and robustness. For the experiments we
chose $\gamma_i = 5.0$.

\subsection{Equality principle-based heuristic (EPBH)}

The second policy is based on the following heuristic: In the best
conditions, every replica should have a similar behavior. Based on his
heuristic, the control variables $\theta_i$ should be as close as
possible to one another. In fact, if the values of $\theta_i$ converge
to a single value, this means that the traffic is routed so that each
replica can serve the same percentage of optional content, i.e., the
most powerful replica is receiving more traffic with respect to the
least powerful one. The distribution should ideally allow to converge
to a value that is maximum with the entire pool of requests received
by the application. This approach therefore selects weights that would
encourage the control variables $\theta_i$ to converge to a single
value.

The policy computes a potential weight $\tilde{w}_i(k+1)$
\begin{equation}
  \tilde{w}_i(k+1) = w_i(k) + \gamma_e e_i(k)
\label{eq:equal-thetas}
\end{equation}
where
$$e_i(k)=\left[ \theta_i (k) - \cfrac{\sum_j \theta_j (k) }{n} \right]$$
and $\gamma_e$ is a non-zero positive parameter of the algorithm which
accounts for how fast the algorithm should converge. For the
experiments we chose $\gamma_e = 0.025$. The weights are simply
modified proportionally to the difference between the current control
value and the average control value set by the replicas. Clearly, the
same saturation and normalization described in Equation
\eqref{eq:theta-diff-rescale} have to be applied to the proposed
solution, to ensure that the sum of the weights is equal to one and
that they have positive values --- i.e., that all the incoming traffic
is directed to the replicas and that each replica receives at least
some requests.

By inserting the expression for $\tilde w_i$ from
\eqref{eq:equal-thetas} in the normalization equation
\eqref{eq:theta-diff-rescale} we get
\begin{equation}
  w_i(k+1)=\cfrac{w_i(k)+\gamma_ee_i(k)}{\sum_jw_j(k)+\gamma_e\sum_je_j(k)} .
\label{eq:weights}
\end{equation}

% Since we normalized all weights $w_i(k)$ to sum to one
% ($\sum_jw_j(k)=1$), and
% $$\sum_je_j(k)=\sum_j\left[\theta_j(k)-\frac{\sum_l\theta_l(k)}{n}\right]=0$$
% we get that \eqref{eq:weights} in stationarity is
% $$w_i(k)=w_i(k)+\gamma_ee_i(k)$$
% and the only solution for this is that $e_i(k)=0$, all dimmers are
% equal in stationarity.

\subsection{Optimization based load-balancing (OBLB)}

The third approach is to update the replica weights based on the
solution of an optimization problem, where the objective is to
maximize the quantity denoted by Equation~\eqref{eq:objective}.

In this solution, each replica is modeled as an $M/G/1$ queue with a
Processor Sharing (PS) discipline. The arrival clients are Markovian
(M) processes, the service times have a general (G) distribution and
each replica consists of a single server. According to the processor
sharing discipline, all clients are served immediately after their
arrival and share a fraction of the available computation power.

\str{Explain why M/G/1 and not M/G/n where n would be the number of
  cores available for the replica. Kind of due to the PS discipline,
  but need better explaination. Also, Karl-Erik commented that we
  cannot assume that control people know what an M/G/1 queue
  is. Background should be better explained.}

We use a general distribution for service times to account for the
brownout adaptation. For each replica $i$ we define $f_{S_i}$, the
probability density function for service times $S_i$, as
\begin{align}
  f_{S_i} (t) = (1 - \theta_i) \cdot \mu_i \cdot e^{-\mu_i \cdot t} +
  \theta_i \cdot M_i \cdot e^{-M_i \cdot t} ,
\label{eq:pdf-servicetimes}
\end{align}
where $t$ represents time, $\theta_i$ is the probability of activating
the optional components, $\mu_i$ denotes the service rate for requests
that do not have optional components enabled and $M_i$ is the service
rates for requests where the optional part is computed. Therefore, the
expression~\eqref{eq:pdf-servicetimes} is a mixture of two
exponentially distributed service times corresponding to serving all
clients with or without optional content. It is well known that for $M/G/1$ queues adopting the processor sharing discipline, the mean response times will only depend on the mean of the service time distribution~\cite{kleinrock67},~\cite{sakata71}, with an effective service rate for each replica $i$ given by
\begin{equation}
  \mu_i^* = \left[ E[S_i] \right]^{-1} = \left[ \frac{1-\theta_i}{\mu_i}
    + \frac{\theta_i}{M_i} \right]^{-1} .
  \label{eq:effective-service-rate}
\end{equation}

It is possible to compute the value of the service rate that respects
the replica objective that the mean response times $t_i$ is equal to
the desired value $t_i^*$, that is given to the brownout-compliant
replica.
\begin{equation}
  t_i = \frac{1}{\mu_i^*-\lambda w_i} = t_i^*
  \Longrightarrow \mu_i^* = \frac{1+t_i^*\lambda w_i}{t_i^*} 
  \label{eq:desired-service-rate}
\end{equation}
Combining Equation~\eqref{eq:effective-service-rate} and
\eqref{eq:desired-service-rate}, one can calculate the optimal values
$\theta_i^*$ of $\theta_i$ that would respect the response time
constraints.  Notice that these values are not used in the replicas
and are simply computed by the optimization based load-balancer as the
optimal stationary conditions for the control variables
$\theta_i$. Clearly, one could also think of using these values within
the replicas but in this investigation we want to completely separate
the load-balancing policy and the replicas internal control loops and
treat the internal loop as a fast system, converging to its stationary
conditions before the next load-balancing intervention.

The values $\theta_i^*$ are
\begin{equation}
  \theta_i^* = \frac{M_i \cdot \left( \mu_i t_i^* - 1 -\lambda w_i
      t_i^* \right)}{{\left( 1+\lambda w_i t_i^* \right) \cdot
      \left(\mu_i-M_i \right)}} = \frac{A_i - B_i w_i}{C_i + D_i w_i}.
  \label{eq:optimal-thetas}
\end{equation}
Recalling that $\theta_i$ is the probability of executing the optional
components when producing the response, the values $\theta_i^*$ should
be constrained to belong to the interval $[0, 1]$. Imposing this
condition provides the following inequalities.
\begin{equation}
  \frac{C_i-A_i}{B_i+D_i} \leq w_i \leq \frac{A_i}{B_i}
  \label{eq:constraints-optimal-thetas}
\end{equation}
Using these inequalities as constraints, it is possible to formally
state the optimization problem as
\begin{equation}
\begin{array}{rl}
  \max_{w_i} \;\;\; & J = \sum_i w_i \theta_i = \sum_i w_i  \frac{A_i - B_i w_i}{C_i + D_i w_i} \\
  \textnormal{s.t.} \;\;\; & \sum_i w_i = 1 \\
  & \max (0,\frac{C_i-A_i}{B_i+D_i}) \leq w_i \leq \frac{A_i}{B_i}
\end{array}
  \label{eq:optimization-problem}
\end{equation}

The objective function $J$ can be shown to be concave in $w_i$ and the
constraints are linear. Therefore the optimization problem is itself
concave and can be solved with standard methods \str{Citation
  needed}. We use CVXOPT, a Python solver for convex optimization, to
obtain the values of the weights.

Notice that solving the optimization
problem~\eqref{eq:optimization-problem} guarantees that the best
possible solution is found, but requires a lot of knowledge about the
single replicas. In fact, while other solutions require knowledge only
about the incoming traffic and the control variables for each replica,
the optimization-based solution relies on knowledge of the service
time of requests with and without optional content $M_i$ and $\mu_i$
that might not be available and could require additional computations
to be estimated correctly.

\str{Find out how the optimizer scales when the number of replicas
  increases and comment on that.}
